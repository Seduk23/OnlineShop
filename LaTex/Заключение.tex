\section*{ЗАКЛЮЧЕНИЕ}
\addcontentsline{toc}{section}{ЗАКЛЮЧЕНИЕ}

В рамках данной курсовой работы был проведен всесторонний анализ, проектирование и разработка веб-сайта по продаже образовательных курсов. Работа с фронтендом и бэкендом, выбор подходящих технологий и детальная проработка пользовательского опыта стали ключевыми моментами в успешной реализации проекта.

Основной целью создания данного веб-сайта было предоставление пользователю удобного, интуитивно понятного инструмента для выбора и приобретения образовательных курсов. Анализ рынка и потребностей пользователей позволил определить ключевые функциональные требования, которые были успешно воплощены в разработанном продукте.

Процесс разработки включал в себя стадии проектирования базы данных, разработки серверной части, создания клиентского интерфейса и реализацию механизмов обработки платежей.

В результате проделанной работы был создан не только функциональный, но и привлекательный веб-сайт, предоставляющий удовлетворительный опыт для конечного пользователя. Проект является успешным примером интеграции технологий в образовательную сферу, а полученные знания и навыки будут полезны в будущих проектах в области веб-разработки.

Основные результаты работы:

\begin{enumerate}
\item Проведен анализ предметной области. Выявлена необходимость использовать Python.
\item Разработана концептуальная модель web-сайта. Разработана модель данных системы. Определены требования к системе.
\item Осуществлено проектирование web-сайта. Разработана архитектура серверной части. Разработан пользовательский интерфейс web-сайта.
\item Реализован и протестирован web-сайт. Проведено системное тестирование.
\end{enumerate}

Все требования, объявленные в техническом задании, были полностью реализованы, все задачи, поставленные в начале разработки проекта, были также решены.

