\section{Технический проект}
\subsection{Общая характеристика организации решения задачи}

Необходимо спроектировать и разработать сайт, который должен способствовать успешной продаже курсов по программированию.

Интернет-сайт представляет собой набор взаимосвязанных электронных страниц, которые сгруппированы по разделам, содержащие текстовую, графическую, а также мультимедийную информацию (изображения и пр.). Сайт располагается в Интернете по определенному адресу – доменному имени сайта в виде www.имя\_сайта.ru. Каждая страница web-сайта – это текстовый документ, написанный на языке программирования (HTML, CSS, JavaScript и т.д.).

\subsection{Обоснование выбора технологии проектирования}

На сегодняшний день информационный рынок, поставляющий программные решения в выбранной сфере, предлагает множество продуктов, позволяющих достигнуть поставленной цели – разработки web-сайта.

\subsubsection{Описание используемых технологий и языков программирования}

В процессе разработки web-сайта используются программные средства и языки программирования. Каждое программное средство и каждый язык программирования применяется для круга задач, при решении которых они необходимы.

\subsubsection{Язык программирования Python}

Python – высокоуровневый язык программирования общего назначения с динамической строгой типизацией и автоматическим управлением памятью, ориентированный на повышение производительности разработчика, читаемости кода и его качества, а также на обеспечение переносимости написанных на нём программ. Использование Python позволило мне легко и эффективно разрабатывать серверную часть проекта, обеспечивая стабильную работу и высокую производительность.

\subsubsection{Язык программирования HTML}

\paragraph{Достоинства языка HTML}

Для разработки клиентской части проекта и отображения информации для пользователей, я использовал язык разметки HTML (HyperText Markup Language). HTML - это стандартный язык, используемый для создания веб-страниц, и он играет ключевую роль во взаимодействии пользователей с моим сервером.

Вот некоторые из основных плюсов HTML:
Простота и доступность: HTML - это язык разметки, который легко учить и использовать. Большинство разработчиков, дизайнеров и даже неспециалистов могут создавать и редактировать веб-страницы с помощью HTML.

Универсальность: HTML поддерживается практически всеми браузерами и устройствами, что позволяет создавать веб-страницы, доступные для широкой аудитории.

Семантика: HTML предоставляет разнообразные элементы и атрибуты для описания структуры и содержания веб-страницы. Это позволяет поисковым системам и другим инструментам легко понимать и анализировать содержание страницы, что положительно влияет на SEO (оптимизацию для поисковых систем).

Версионная совместимость: HTML постоянно развивается, и новые версии (например, HTML5) включают в себя новые возможности и элементы. Это обеспечивает совместимость с более ранними версиями HTML и позволяет использовать современные функции.

Встроенные мультимедийные возможности: HTML позволяет вставлять мультимедийные элементы, такие как изображения, аудио и видео, непосредственно в веб-страницы без необходимости использования сторонних плагинов.

Поддержка форм и интерактивности: HTML позволяет создавать веб-формы для сбора данных от пользователей и обеспечивает возможность создания интерактивных элементов, таких как кнопки и ссылки.

\paragraph{Недостатки языка HTML}

Разметка зависит от открытия и закрытия тегов: В HTML необходимо явно открывать и закрывать теги, и неверно закрытые или упущенные теги могут привести к ошибкам в отображении страницы:

<p>Это абзац</p>

<p>Это другой абзац</p>

Глубокая вложенность тегов: В случае слишком глубокой вложенности тегов HTML-код может стать сложным для понимания и поддержки:

<div>

<p>

<span>Текст внутри span</span>

</p>

</div>

Ограниченный контроль стилей: HTML ограничивается только структурой и содержанием веб-страницы.

\subsection{Диаграмма компонентов и схема обмена данными между файлами компонента}

Диаграмма компонентов описывает особенности физического представления разрабатываемой системы. Она позволяет определить архитектуру системы, установив зависимости между программными компонентами, в роли которых может выступать как исходный, так и исполняемый код. Основными графическими элементами диаграммы компонентов являются компоненты, интерфейсы, а также зависимости между ними. На рисунке \ref{comp:image} изображена диаграмма компонентов для проектируемой системы. Она включает в себя сервер с операционной системой, на которой установлена система управления содержимым, включающая в себя базу данных и интерфейс. Помимо этого на диаграмме изображен клиентский компьютер с операционной системой, на которой установлен браузер.

\begin{figure}[ht]
\center{\includegraphics[width=1\linewidth]{comp}}
\caption{Диаграмма компонентов}
\label{comp:image}
\end{figure}

Любой компонент должен быть вызван в сценарии страницы web-сайта. Web-страница передает данные компоненту в момент вызова последнего.

На рисунке \ref{data:image} представлена схема обмена данными между сценариями компонента при вызове компонента на странице сайта.

\begin{figure}[ht]
\center{\includegraphics[width=1\linewidth]{data}}
\caption{Диаграмма компонентов}
\label{data:image}
\end{figure}

При вызове компонента в сценарии web-страницы указываются значения параметров компонента, которые далее посредством массива \$arParams передаются в сценарий файла component.php.

В сценарии файла component.php посредством метода \linebreak IncludeComponentTemplate класса CBitrixComponent происходит вызов одного из шаблонов компонента. Id шаблона также определяется в сценарии страницы web-приложения и неявно для разработчика передается указанный выше метод. Подключается сценарий файла template.php одного из шаблонов, в который передается, возможно, измененный в сценарии component.php массив \$arParams и, также, сформированный в сценарии component.php массив \$arResult. Оба этих массива доступны также и в файле result\_modifier.php, который подключается перед подключением файла template.php. 

Работа компонента заканчивается в момент завершения работы сценария файла component.php, т.е. возможно выполнить действия уже после подключения шаблона. Однако, если массив \$arResult будет изменен в сценарии шаблона, в сценарий файла компонента component.php измененные данные переданы не будут.

\subsection{Диаграмма размещения}

Диаграмма размещения (рис.~\ref{place:image}) отражает физические взаимосвязи между программными и аппаратными компонентами системы.

\vspace{-8mm} % чтобы убрать пустую строку, которая осталась после переноса рисунка на следующую страницу
\begin{figure}[ht]
\center{\includegraphics[width=0.57\linewidth]{place}}
\caption{Диаграмма размещения. Не помещается на страницу. Очень длинный заголовок}
\label{place:image}
\end{figure}

Она является хорошим средством для показа маршрутов перемещения объектов и компонентов в распределенной системе.

В таблице \ref{ssevsws:table} приведен пример использования пакета xltabular с автоматическим расчетом ширины столбца.

\begin{xltabular}{\textwidth}{|c|X|X|}
	\caption{Сравнение протоколов SSE и WebSocket\label{ssevsws:table}}\\ \hline
	~  & \centrow  SSE & \centrow WebSocket \\ \hline
	\endfirsthead
	\continuecaption{Продолжение таблицы \ref{ssevsws:table}}
	~ & \centrow SSE & \centrow WebSocket \\ \hline 
	\finishhead
	Направленность & 
	Однонаправленный, полудуплексный: данные посылает только сервер & 
	Двунаправленный, полнодуплексный: и сервер, и клиент могут обмениваться сообщениями \\ \hline 
	Соединение  & HTTP & WS \\ \hline 
	Тип данных & Только текст & Бинарные и текстовые данные \\ \hline 
	Доп. возможности & Встроенный механизм идентификаторов событий и переподключения & Переподключение и идентификация события реализуются на стороне приложения
\end{xltabular}

\subsection{Содержание информационных блоков. Основные сущности}

Проанализировав требования, можно выделить шесть основных сущностей:
\begin{itemize}
\item "<Новости">;
\item "<Продукция">;
\item "<Услуги">.
\end{itemize}

В состав сущности "<Новости"> можно включить атрибуты, представленные в таблице \ref{news:table}.

\begin{xltabular}{\textwidth}{|l|l|p{1.7cm}|X|}
	\caption{Атрибуты сущности "<Новости">\label{news:table}}\\ \hline
	\centrow Поле & \centrow Тип & \centrow Обяза\-тельное & \centrow Описание \\ \hline
	\thead{1} & \thead{2} & \centrow 3 & \centrow 4 \\ \hline
	\endfirsthead
	\continuecaption{Продолжение таблицы \ref{news:table}}
	\thead{1} & \thead{2} & \centrow 3 & \centrow 4 \\ \hline
	\finishhead
	\_id & ObjectId & true & Уникальный идентификатор \\ \hline 
	head & String & true & Заголовок новости \\ \hline 
	short & String & false & Аннотация к новости \\ \hline 
	createdAt & Date & true & Время создания новости \\ \hline 
	author & String & false & Автор новости \\ \hline 
	content & String & true & Текст новости \\ \hline 
	views & Integer & true & Количество просмотров новости зарегистрированными пользователями
\end{xltabular}

Пример использования различных типов столбцов представлен в таблице \ref{prod:table}. Рекомендуется использовать пакет xltabular для создания таблиц.

\begin{xltabular}{\textwidth}{|R|C{2.5cm}|l|T|}
	\caption{Атрибуты  сущности "<Новости разметки в LaTeX"> с использованием различных типов столбцов и многострочным заголовком\label{prod:table}}\\ \hline
	\centrow Поле & \centrow Тип & \centrow Обязательное & \centrow Описание \\ \hline
	\centrow 1 & \centrow 2 & \thead{3} & \centrow 4 \\ \hline
	\endfirsthead
	\continuecaption{Продолжение таблицы \ref{prod:table}}
	\centrow 1 & \centrow 2 & \thead{3} & \centrow 4 \\ \hline
	\finishhead
	\_id & ObjectId & true & Уникальный идентификатор \\ \hline 
	head & String & true & Заголовок новости \\ \hline 
	short & String & false & Аннотация к новости \\ \hline 
	createdAt & Date & true & Время создания новости \\ \hline 
	author & String & false & Автор новости \\ \hline 
	content & String & true & Текст новости \\ \hline 
	views & Integer & true & Количество просмотров новости зарегистрированными пользователями
\end{xltabular}

В системе предусмотрен внутренний механизм связи между разделами и элементами информационных блоков, поэтому введения дополнительных идентификаторов при реализации связей между сущностями не предполагается.

Экземпляры сущностей реализуются в информационных блоках посредством элементов, атрибуты сущности – посредством полей и свойств элемента. 
