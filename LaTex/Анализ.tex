\section{Анализ предметной области}
\subsection{Характеристика предприятия и его деятельности.}

В современном образовательном ландшафте онлайн магазины курсов представляют собой значительное изменение в том, как люди получают знания и развивают свои профессиональные навыки. Этот раздел посвящен более глубокому анализу предметной области, охватывая ключевые аспекты функционирования онлайн магазинов курсов и их влияние на образовательные процессы.

1. Развитие онлайн образования: тенденции и динамика

Первоначальный взлет онлайн образования произошел в контексте постоянного роста технологических возможностей. Эволюция интернета и цифровых технологий обеспечила создание разнообразных онлайн платформ, предлагающих курсы по самым разным предметам. Важным аспектом этой тенденции является расширение доступа к образованию, снижение географических барьеров и предоставление возможности для обучения в удобное время.

2. Преимущества онлайн магазинов курсов

Анализируя функционал онлайн магазинов курсов, мы обращаем внимание на их преимущества. Гибкость в выборе предметов, доступность 24/7, индивидуальный темп обучения и множество форматов обучения — все эти факторы делают онлайн курсы привлекательными для разнообразных аудиторий, начиная от студентов до профессионалов, стремящихся обновить свои знания.

3. Вызовы и перспективы

Неоспоримые преимущества онлайн магазинов курсов сопровождаются вызовами. Отсутствие физической интеракции, проблемы с оценкой студенческого прогресса и необходимость постоянного обновления контента — все это становится объектом внимания образовательных исследований. Однако, несмотря на эти вызовы, онлайн магазины курсов имеют перспективы дальнейшего развития, особенно в условиях постоянно меняющегося мира труда.

4. Влияние на традиционное образование

Интересен также вопрос о влиянии онлайн магазинов курсов на традиционные образовательные учреждения. Как они адаптируются к конкуренции, предлагают ли свои курсы в онлайн формате, и какие изменения происходят в подходах к обучению в классе?

Все эти аспекты предметной области требуют более детального анализа для полного понимания того, как онлайн магазины курсов становятся неотъемлемой частью современной образовательной экосистемы. Дальнейшее исследование позволит выявить тенденции, прогнозировать будущие изменения и предоставить практические рекомендации для улучшения качества образования в онлайн формате.
\subsection{Аддитивные технологии, их классификация}

Основное преимущество АТ состоит в том, что прототип создается за один прием, а исходными данными для него служит геометрическая модель детали. В итоге отпадает необходимость в планировании последовательности технологических процессов, специальном оборудовании для обработки материалов, транспортировке от станка к станку и т. д.

Экструзионная печать. Включает такие методы, как послойное наплавление и многоструйная печать.

Стереолитография. Стереолитографические принтеры используют специальные жидкие материалы, называемые "<фотополимерными смолами">.

Ламинирование. Слои материала наклеиваются друг на друга и обрезаются по контурам цифровой модели с помощью лазера или лезвия. 
