\section*{ВВЕДЕНИЕ}
\addcontentsline{toc}{section}{ВВЕДЕНИЕ}

В современном информационном обществе, под воздействием стремительного развития технологий, образование приобретает новые измерения. Студенты и профессионалы ищут удобные и эффективные способы обучения, а интернет становится неотъемлемой частью этого процесса. В этом контексте, онлайн-образование приобретает все большую популярность, а онлайн магазины курсов выступают важным звеном в современной образовательной парадигме.

Цель данной курсовой работы — исследовать и проанализировать концепцию онлайн магазина курсов, рассмотреть его преимущества и вызовы, а также выявить влияние данного формата образования на процесс обучения и доступ к знаниям. Путем анализа существующих платформ и тенденций в данной области, мы стремимся глубже понять, как онлайн магазины курсов взаимодействуют с современной образовательной средой и в чем заключаются их основные особенности.

Развитие технологий и их влияние на образование предоставляют уникальные возможности для инноваций в обучении. Онлайн магазины курсов являются проявлением этих изменений, предоставляя гибкость и доступность обучения, которые ранее казались невозможными. В нашей работе мы попытаемся рассмотреть различные аспекты этого феномена, выявить его преимущества и недостатки, а также оценить перспективы развития онлайн образования через призму онлайн магазинов курсов.

Главной задачей профессионально построенного сайта является превращение посетителя, зашедшего на сайт, в потенциального клиента.

\emph{Цель настоящей работы} – разработка web-сайта онлайн-магазина для привлечения новой аудитории, увеличения заказов, рекламы продукции и услуг компании. Для достижения поставленной цели необходимо решить \emph{следующие задачи:}
\begin{itemize}
\item провести анализ предметной области;
\item разработать концептуальную модель web-сайта;
\item спроектировать web-сайт;
\item реализовать сайт средствами web-технологий.
\end{itemize}

\emph{Структура и объем работы.} Отчет состоит из введения, 4 разделов основной части, заключения, списка использованных источников, 2 приложений. Текст выпускной квалификационной работы равен \formbytotal{page}{страниц}{е}{ам}{ам}.

\emph{Во введении} сформулирована цель работы, поставлены задачи разработки, описана структура работы, приведено краткое содержание каждого из разделов.

\emph{В первом разделе} на стадии описания технической характеристики предметной области приводится сбор информации о деятельности компании, для которой осуществляется разработка сайта.

\emph{Во втором разделе} на стадии технического задания приводятся требования к разрабатываемому сайту.

\emph{В третьем разделе} на стадии технического проектирования представлены проектные решения для web-сайта.

\emph{В четвертом разделе} приводится список классов и их методов, использованных при разработке сайта, производится тестирование разработанного сайта.

В заключении излагаются основные результаты работы, полученные в ходе разработки.

В приложении А представлен графический материал.
В приложении Б представлены фрагменты исходного кода. 
